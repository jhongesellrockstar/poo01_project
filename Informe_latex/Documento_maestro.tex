\documentclass[12pt,a4paper]{book}
\usepackage[utf8]{inputenc}
\usepackage[T1]{fontenc}
\usepackage[spanish,es-tabla]{babel}

% \usepackage{hyphenat}
% \hyphenation{apren-di-zaje pro-fun-do}

\usepackage[left=2.5cm,right=2cm,top=2cm,bottom=1.7cm,includefoot,includehead,headheight=16pt]{geometry}

\usepackage{lipsum}

% Bibliografía
\usepackage[style=authoryear]{biblatex}
\usepackage{csquotes}
\addbibresource{bibliografia.bib}
% \usepackage[authoryear,round]{natbib}
% \bibliographystyle{unsrtnat}

% Control sobre los márgenes
\setlength{\topmargin}{-1cm}
\setlength{\textheight}{24cm}
\setlength{\linewidth}{3cm}

% Estilo
\usepackage{amsmath,amssymb,amsfonts,stackrel,hyperref}
% \numberwithin{equation}{section}
\usepackage{graphicx,xcolor}
% Colores de las paletas de identidad corporativa de la UNED
\definecolor{UNED}{RGB}{0, 83, 62}
\definecolor{UNED_medium}{RGB}{66, 117, 98}
\definecolor{Apple}{RGB}{116, 159, 76}
\definecolor{Apple_medium}{RGB}{138, 172, 93}
\definecolor{Blue}{RGB}{92, 110, 177}
\definecolor{Blue_medium}{RGB}{118, 131, 189}
\definecolor{Tangerine}{RGB}{215, 111, 71}
\definecolor{Tangerine_medium}{RGB}{221, 137, 100}
\definecolor{Strawberry}{RGB}{218, 82, 104}
\definecolor{Strawberry_medium}{RGB}{200, 115, 132}
\definecolor{Raspberry}{RGB}{144, 33, 74}
\definecolor{Raspberry_medium}{RGB}{142, 77, 96}

\usepackage{setspace}
\renewcommand{\baselinestretch}{1.2}

\hypersetup{
    colorlinks,
    linkcolor=blue,
    citecolor=blue,
    urlcolor=blue
}

% Tablas
\usepackage{tabularx,multirow,rotating,longtable}

% Código
\usepackage{listings} 
\renewcommand{\lstlistingname}{Código}
\usepackage[mathscr]{euscript}
\lstdefinestyle{estiloMUICD}{
    backgroundcolor=\color{black!5!white},
    commentstyle=\color{green!60!black},
    keywordstyle=\color{blue},
    numberstyle=\footnotesize,
    stringstyle=\color{black!40!white}\ttfamily,
    basicstyle=\small\ttfamily,
    breakatwhitespace=false,
    breaklines=true,
    captionpos=b,
    keepspaces=true,
    numbers=left,
    numbersep=8pt,
    showspaces=false,
    showstringspaces=false,
    showtabs=false,
    tabsize=2
}
\lstset{style=estiloMUICD}


% Tikz
\usepackage{tikz}
\usetikzlibrary{arrows}

% Encabezados
\usepackage{fancyhdr}
\fancyhf{}
\fancyhead{}
\pagestyle{fancy}
\fancyhead[LE,RO]{\thepage}
\fancyhead[LO, RE]{\leftmark}
\renewcommand{\chaptermark}[1]{\markboth{#1}{}}
\renewcommand{\sectionmark}[1]{\markright{#1}}
\pagestyle{empty}

% Glosarios y nomenclaturas
\usepackage[toc,nonumberlist]{glossaries}
\makeglossaries

\newglossaryentry{poo}{
    name={POO},
    description={Paradigma de Programaci\'on Orientada a Objetos empleado para organizar la l\'ogica de la aplicaci\'on en clases y m\'odulos}
}

\newglossaryentry{sqlite}{
    name={SQLite},
    description={Motor de base de datos relacional embebido, utilizado en la versi\'on ligera del sistema para evitar dependencias de servidor}
}

\newglossaryentry{tkinter}{
    name={Tkinter},
    description={Biblioteca gr\'afica est\'andar de Python empleada para construir la interfaz de escritorio del sistema}
}

\newglossaryentry{blacklist}{
    name={Lista negra},
    description={Registro colaborativo de personas o establecimientos reportados por maltrato o malas pr\'acticas hacia mascotas}
}

\newglossaryentry{mapa}{
    name={Mapa interactivo},
    description={Componente basado en \texttt{tkintermapview} que permite ubicar incidentes y cl\'inicas en Lima y Callao}
}



\begin{document}
% \glsaddall
\frontmatter
\begin{titlepage}
\centering
        \includegraphics[height=3.5cm]{imagenes/logo_informatica_s.png}\\
        \vspace{0.25cm}
        {\Large \textsc{Universidad Nacional del Callao}}\\
        \vspace{0.1cm}
        {\large Facultad de Ingeniería Industrial y de Sistemas}\\
        \vspace{0.25cm}
        {\large Escuela Profesional de Ingeniería de Sistemas}\\
        \vspace{0.9cm}
    \begin{spacing}{1.6}
        {\textsc{\Huge Patitas Seguras: plataforma de alerta, ranking y adopción}}\\
        {\Large Diseño modular con base de datos SQLite y evidencias para Lima y Callao}
    \end{spacing}
        \vfill
        {\Large Estudiante 1 (código 1234)\\Estudiante 2 (código 1235)\\Estudiante 3 (código 1236)\\Estudiante 4 (código 1237)}\\
        \vspace{0.5cm}
        \begin{tabular}{ll}
        \large Docente: & \large Reinoso Palacios Artemio Rub\'en\\
        \vspace{0.3cm}
        \large Curso: & \large Programaci\'on Orientada a Objetos\\
        \vspace{0.3cm}
        \large Modalidad: & \large Proyecto aplicado
        \end{tabular}

        \vfill
        {\Large Informe del proyecto de curso}\\
        \vspace{0.3cm}
        {\large Ciclo académico 2024-II}\\
        \vspace{0.25cm}
        Callao, Per\'u
\end{titlepage}

\cleardoublepage{}

\begin{flushright}
\section*{Agradecimientos}
[Quisiera dedicar \ldots]
\par\end{flushright}    



\cleardoublepage{}
\noindent \begin{center}
\section*{Resumen}
\par\end{center}

\lipsum[1-2]

\vspace{0.5cm}
\begin{flushleft}
\textbf{Palabras clave}: palabra1, palabra2, palabra3
\end{flushleft}

\cleardoublepage{}
\selectlanguage{english}%
\noindent \begin{center}
\section*{Abstract}
\par\end{center}

\lipsum[1-2]

\vspace{0.5cm}
\begin{flushleft}
\textbf{Keywords}: keyword1, keyword2, keyword3
\end{flushleft}
\selectlanguage{spanish}%

%\newpage
\printglossary


\tableofcontents
\listoffigures
\renewcommand{\listtablename}{Índice de tablas}
\listoftables


%\newpage
\mainmatter
\pagestyle{fancy}

\chapter{Introducci\'on}\label{chap:introduccion}

\section{Motivaci\'on}
Lima Metropolitana y el Callao concentran la mayor poblaci\'on urbana del pa\'is y registran problemas crecientes de bienestar animal, incluyendo abandono, maltrato y brotes de zoonosis asociados a la densidad de perros y gatos callejeros \citep{ley30407,serfor2023diagnostico}. Los estudiantes de Ingenier\'ia de Sistemas (c\'odigos 1234, 1235, 1236 y 1237) asumimos el reto de proponer una herramienta tecnol\'ogica que apoye la denuncia ciudadana y la toma de decisiones sobre atenci\'on veterinaria.

\section{Propuesta y objetivos}
El proyecto \textit{Patitas Seguras} consolida tres l\'ineas de acci\'on: (1) una lista negra colaborativa para maltratadores y veterinarias sancionadas, (2) un tablero de adopciones y mascotas perdidas, y (3) un ranking comunitario de servicios veterinarios para elevar la transparencia. El presente informe describe el dise\~no modular del software y presenta un segundo despliegue basado en SQLite que elimina la dependencia de un servidor corporativo, manteniendo la funcionalidad prevista en el archivo \texttt{main.py}. Los objetivos espec\'ificos son documentar la arquitectura, las decisiones de dise\~no y las oportunidades de mejora en futuras iteraciones.

\section{Estructura del documento}
El documento se organiza en cinco cap\'itulos: el Cap.\ref{chap:introduccion} expone la motivaci\'on y alcance; el Cap.\ref{chap:estadoarte} sintetiza la literatura y normativa relevante; el Cap.\ref{chap:metodologia} detalla los materiales, m\'odulos y base de datos; el Cap.\ref{chap:resultados} muestra resultados preliminares y capturas funcionales; finalmente, el Cap.\ref{chap:conclusiones} presenta conclusiones y trabajo futuro.

\chapter{Estado del arte y contexto regional}\label{chap:estadoarte}

\section{Normativa y planes nacionales}
La Ley N\textdegree{} 30407 establece el marco peruano de protecci\'on y bienestar animal, obligando a denunciar el maltrato y promoviendo la tenencia responsable \citep{ley30407}. Complementariamente, el Servicio Nacional Forestal y de Fauna Silvestre (SERFOR) public\'o en 2023 un diagn\'ostico que subraya la falta de registros unificados de mascotas y la necesidad de articular a gobiernos locales \citep{serfor2023diagnostico}. Estas referencias justifican el componente de listas negras y la georreferenciaci\'on de reportes.

\section{Situaci\'on en Lima y Callao}
Estudios locales indican un aumento de denuncias por maltrato y sobrepoblaci\'on canina en distritos populosos de Lima y el Callao, lo que incrementa el riesgo de rabia y otras zoonosis \citep{ops2022rabia}. La Municipalidad Metropolitana de Lima emiti\'o ordenanzas que promueven la identificaci\'on de mascotas, pero la adopci\'on de herramientas digitales sigue siendo limitada \citep{ordenanzaLima}. En este escenario, una aplicaci\'on de escritorio con base SQLite facilita pilotos en laboratorios acad\'emicos y cabinas municipales sin conectividad permanente.

\section{Referentes tecnol\'ogicos}
Existen iniciativas de mapeo ciudadano basadas en \textit{open data} y sistemas ligeros (por ejemplo, dashboards de dengue y COVID-19) que inspiran el uso de \texttt{tkintermapview} y esquemas relacionales compactos \citep{whoZero2023}. La propuesta adopta patrones de arquitectura MVC y refuerza la trazabilidad mediante cat\'alogos de usuarios y auditor\'ia de votos.

\chapter{Materiales y m\'etodos}\label{chap:metodologia}

\section{Recursos de software}
Se emple\'o Python 3.11 con las bibliotecas \texttt{tkinter} para la interfaz, \texttt{Pillow} para manejo de im\'agenes y \texttt{tkintermapview} para mostrar ubicaciones. El prototipo original usa \texttt{pyodbc} contra SQL Server (archivo \texttt{main.py}); en este informe se documenta un segundo script con \texttt{sqlite3} que reside en la carpeta \texttt{Code} y genera un archivo local \texttt{patitas.db} para facilitar pruebas sin red.

\section{Dise\~no de base de datos}
La base incluye las tablas \texttt{Usuarios}, \texttt{BlacklistMaltratadores}, \texttt{BlacklistVeterinarias}, \texttt{RankingVeterinarias} y \texttt{VotosVeterinarias}. Cada entidad incorpora campos de trazabilidad (fechas y referencias) y permite c\'alculos de promedios de estrellas. Se inicializan datos semilla con casos ficticios en Callao y Lima para validar la navegaci\'on.

\section{Arquitectura y modularidad}
El c\'odigo se organiza en clases: la clase principal extiende \texttt{tk.Tk}, gestiona el estado de sesi\'on y delega en funciones espec\'ificas para men\'us, formularios y consultas. Se implementaron funciones reutilizables para crear barras de navegaci\'on, lienzos con desplazamiento y manejo de im\'agenes. El nuevo script desacopla la configuraci\'on de conexi\'on mediante la funci\'on \texttt{get\_connection()}, que garantiza la creaci\'on de tablas y el cierre seguro de la conexi\'on a SQLite.

\section{Flujo de trabajo}
El flujo sigue cuatro pasos: (1) el usuario se registra o inicia sesi\'on; (2) se listan tarjetas de reportes de maltrato o veterinarias con filtros; (3) se presentan mapas y datos de contacto; y (4) se permite votar o registrar nuevas entradas seg\'un el rol. Las interacciones se validan con mensajes amigables y se recalculan promedios en cada inserci\'on de voto.

\chapter{Resultados}\label{chap:resultados}

\section{Prototipo original}
El archivo \texttt{main.py} conserva el dise\~no de interfaz con paleta de colores consistentes, men\'u principal y m\'odulos de login, listas negras y ranking. Las consultas dependen de SQL Server y actualmente requieren ajustes de cadena de conexi\'on para funcionar fuera del entorno acad\'emico.

\section{Script alternativo con SQLite}
Se implement\'o \texttt{Code/patitas\_sqlite.py} como clon funcional que crea y puebla autom\'aticamente la base \texttt{patitas.db}. El script mantiene los flujos de inicio de sesi\'on, registro de usuarios, altas en listas negras y calificaci\'on de veterinarias. La funci\'on \texttt{seed\_data()} agrega dos entradas de prueba (Callao y Lima) para validar el mapa y los promedios.

\section{Hallazgos y validaci\'on}
La modularidad permiti\'o reutilizar componentes de interfaz y aislar la l\'ogica de base de datos. El cambio a SQLite reduce fricci\'on de despliegue y facilita demostraciones sin credenciales corporativas. Pendientes futuros incluyen incorporar formularios completos de adopci\'on y p\'erdida, as\'i como automatizar la carga de evidencias (fotos y documentos) por distrito.

\chapter{Conclusiones y trabajos futuros}\label{chap:conclusiones}

El proyecto \textit{Patitas Seguras} integra denuncia ciudadana, georreferenciaci\'on y reputaci\'on colaborativa para mejorar el bienestar animal en Lima y Callao. La versi\'on en SQLite demuestra que es posible ejecutar el prototipo en entornos aislados manteniendo los flujos previstos en el desarrollo original.

Como trabajo futuro se plantea: (1) completar los formularios de adopci\'on y mascotas perdidas, (2) incorporar validaci\'on de archivos multimedia y resguardo en la base, (3) sumar anal\'iticas de calor por distrito y (4) exponer servicios web REST para integraci\'on con aplicaciones m\'oviles.



\newpage


% \bibliography{bibliografia}
\printbibliography
\addcontentsline{toc}{chapter}{Bibliografía y referencias} %Para 
% \nocite{*}
\glsaddall


% Apéndices
\appendix
\chapter{Ejemplos en \LaTeX}

Este primer apéndice presenta ejemplos en \LaTeX de cómo incluir referencias, citas bibliográficas, figuras, tablas o código. Este apéndice se deberá eliminar de la memoria antes de entregar el trabajo. 

\section{Referencias, citas y bibliografía}
\subsection{Referencias}
Para poder referenciar un elemento dentro de la memoria hay que marcarlo con una etiqueta (\verb|\label{[id]}|) que lo identifique inequívocamente. Es habitual utilizar identificadores  representativos, por ejemplo, para marcar la introducción podemos utilizar una etiqueta como \verb|\label{chap:introduccion}|

Una vez marcado el elemento (capítulo, sección, figura, tabla, \ldots) en \LaTeX, utilizaremos el comando \verb|ref| indicando a qué etiqueta queremos referenciar (\verb|\ref{[id]}|). Por ejemplo, de esta manera podemos referenciar al capítulo de la introducción con \ref{chap:introduccion}. Podemos acompañarlo de un texto como ``\ldots como se vió en el capítulo~\ref{chap:introduccion}\ldots'' o bien utilizar el comando \verb|\autoref{[i]}|, que incluiría el tipo del elemento y aparecería en el texto como \autoref{chap:introduccion}, o incluso referenciarlo por el nombre del elemento con \verb|\nameref{[id]}|, lo que se mostraría como \nameref{chap:introduccion}. 

\subsection{Bibliografía}
Para añadir una cita bibliográfica al documento tendremos que asegurarnos que en el fichero de bibliografía (bibliografía.bib) se encuentre la entrada bibliográfica correspondiente. Una vez que tengamos nuestra entrada bibliográfica utilizaremos el comando de cita de \LaTeX indicando el identificador de la referencia bibliográfica, por ejemplo: 
\verb|\cite{aikg}|, que se se visualizaría como \cite{aikg}. 

Cuando queramos citar más de un trabajo en un mismo punto del documento se deberán añadir a la cita separados por comas, por ejemplo: \verb|\cite{chen_2018, Qin_2020}|, que se visualizaría como \cite{chen_2018, Qin_2020}. En estos casos es habitual añadir las citas en orden cronológico de más antigua a más reciente.

Tal y como está configurada esta plantilla, por defecto el comando \verb|\cite| realizará una cita textual (\verb|\citet| si se activa el paquete \verb|natbib| en lugar de \verb|biblatex|), es decir, la cita forma parte del texto. Este tipo de citas se suelen realizar en frases como: 

``Algunos ejemplos de grafos de conocimiento se presentan en \cite{ji2020survey}'' 

Cuando la cita no forma parte del texto si no que se utiliza para reforzar una afirmación realizada en una frase, se debe utilizar una cita entre paréntesis (o corchetes). Para ello se usa el comando \verb|\parencite| en biblatex o \verb|\citep| en natbib, por ejemplo:

``Existe una gran variedad de aplicaciones de los grafos de conocimiento \parencite{ji2020survey}.''


\section{Figuras}
Para añadir una figura al documento utilizaremos el entorno figure, indicando la posición que debe tener dicha figura en el documento (h: here, t: top, b: botom; p: page), se recomienda en lo posible evitar de ``!'' que ignora todos los ajustes de los parámetros. El orden en el que se indiquen cada una de las opciones se tendrá en cuenta para colocar la figura, es decir si se indicase el orden [htbp] la figura primero se intentará colocar en el lugar que ocupa en el documento, si no se puede se intentará colocar al inicio (top) de la página, en caso que tampoco sea posible se intentará colocar al final (bottom) de la página y, por último en caso que no sea posible ninguna de las anteriores se colocará al inicio de una página nueva.

\begin{figure}[ht]
\begin{centering}
\includegraphics[width=0.5\columnwidth]{imagenes/logo_informatica.png}
\par\end{centering}

\caption[Ejemplo de figura]{Esta figura tiene una descripción al pie muy larga, por lo que añadiremos un título breve utilizando para ello los corchetes tras el comando \textbackslash caption. La etiqueta de la figura (label) se incluirá al incio del flotante de la figura para que cualquier referencia cruzada (ref) a la misma lleve al inicio del flotante.\label{fig:Ejemplo-de-figura}}
\end{figure}

\section{Tablas}
Para añadir una tabla se utilizará el entorno \verb|table|, indicando al incio de la tabla el título de la misma utilizando el \verb|caption|. Un ejemplo puede verse en la Tabla~\ref{tab:Ejemplo-de-tabla}.

\begin{table}[ht]

\centering
\caption[Ejemplo de tabla]{Esta tabla presenta un ejemplo con tres columnas y formato formal.\label{tab:Ejemplo-de-tabla}}
% Alineación de cada columna: l = left; c = center; r = right
\begin{tabular}[t]{lccccc}
\hline
Model & Accuracy & Precision & Recall & F1-score & AUC\\ % La doble barra indica un salto de línea
\hline
Modelo 1 & $0.33$ & $0.75$ & $0.72$ & $0.42$ & $0.21$ \\
Modelo 2 & $0.01$ & $0.63$ & $0.60$ & $0.50$ & $0.10$ \\
Modelo 3 & $0.03$ & $0.93$ & $0.33$ & $0.04$ & $0.42$ \\
\hline
\end{tabular}
\end{table}



\section{Código}
Para añadir código en la memoria utilizaremos el paquete \verb|listing|, que permite mostrar código formateado en diversos lenguajes (Java, Python, C, \ldots). 

\begin{lstlisting}[language=Python, caption={Título del fragmento de código}, captionpos=b]
import numpy as np
    
def incmatrix(genl1,genl2):
    m = len(genl1)
    n = len(genl2)
    M = None #to become the incidence matrix
    VT = np.zeros((n*m,1), int)  #dummy variable
    
    #compute the bitwise xor matrix
    M1 = bitxormatrix(genl1)
    M2 = np.triu(bitxormatrix(genl2),1) 

    for i in range(m-1):
        for j in range(i+1, m):
            [r,c] = np.where(M2 == M1[i,j])
            for k in range(len(r)):
                VT[(i)*n + r[k]] = 1;
                VT[(i)*n + c[k]] = 1;
                VT[(j)*n + r[k]] = 1;
                VT[(j)*n + c[k]] = 1;
                
                if M is None:
                    M = np.copy(VT)
                else:
                    M = np.concatenate((M, VT), 1)
                
                VT = np.zeros((n*m,1), int)
    
    return M
\end{lstlisting}

\section{Listas}
Como en cualquier procesador de textos debemos diferenciar dos tipos de listas. Las listas no numeradas (o de viñetas) se definirán mediante entornos \verb|itemize|. 

\begin{itemize}
    \item Elemento no numerado
    \item Elemento no numerado 
    \item Elemento no numerado
\end{itemize}

Las listas numeradas se definirán mediantes entornos \verb|enumerate|. En ambos casos, cada elemento de la lista se genera utilizando el comando \verb|item|. 

\begin{enumerate}
    \item Elemento 1 \\ Con el comando \verb|\\| podemos insertar saltos de línea sin cambiar de párrafo.
    \item Elemento 2. Con el comando \verb|\textbf| podemos \textbf{enfatizar con negrita} un texto y con el comando \verb|\textit| podemos \textit{enfatizar con itálica} un texto.
    \item Elemento 3
\end{enumerate}

\section{Ecuaciones}
Cuando una ecuación va a ser referenciada desde el texto, en principio más de una vez, será necesario nombrar o numerar dicha ecuación (ver \autoref{Eq:ejemplo_ecuacion1}). Para añadir una ecuación numerada utilizaremos el entorno \verb|equation|.

\begin{equation}\label{Eq:ejemplo_ecuacion1}
\left(\begin{array}{cccc}w_{1}, & w_{2}, & ... & ,w_{n}\end{array}\right)
\left(\begin{array}{c}x_{1}\\x_{2}\\\vdots\\x_{n}\end{array}\right)+b=0
\rightarrow\mathbf{w^{\mathsf{t}}}\mathbf{x}+b=0
\end{equation}

Cuando la ecuación no va a ser referenciada desde el texto, podemos utilizar la versión de entorno no numerada \verb|equation*|

\begin{equation*}
\label{Eq:ejemplo_ecuacion2}
\dfrac{1}{2}\parallel\mathbf{w}\parallel=\dfrac{1}{2}\sqrt{\sum w_{i}^{2}}
\end{equation*}

Es también posible añadir ecuaciones en línea con el texto, para ello se debe incluir la expresión matemática en \LaTeX encerrada entre símbolos de dólar. Por ejemplo, la ecuación anterior se podría representar también en línea: $\dfrac{1}{2}\parallel\mathbf{w}\parallel=\dfrac{1}{2}\sqrt{\sum w_{i}^{2}}$.

\end{document}