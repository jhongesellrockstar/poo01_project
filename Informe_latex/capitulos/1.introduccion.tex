\chapter{Introducci\'on}\label{chap:introduccion}

\section{Motivaci\'on}
Lima Metropolitana y el Callao concentran la mayor poblaci\'on urbana del pa\'is y registran problemas crecientes de bienestar animal, incluyendo abandono, maltrato y brotes de zoonosis asociados a la densidad de perros y gatos callejeros \citep{ley30407,serfor2023diagnostico}. Los estudiantes de Ingenier\'ia de Sistemas (c\'odigos 1234, 1235, 1236 y 1237) asumimos el reto de proponer una herramienta tecnol\'ogica que apoye la denuncia ciudadana y la toma de decisiones sobre atenci\'on veterinaria.

\section{Propuesta y objetivos}
El proyecto \textit{Patitas Seguras} consolida tres l\'ineas de acci\'on: (1) una lista negra colaborativa para maltratadores y veterinarias sancionadas, (2) un tablero de adopciones y mascotas perdidas, y (3) un ranking comunitario de servicios veterinarios para elevar la transparencia. El presente informe describe el dise\~no modular del software y presenta un segundo despliegue basado en SQLite que elimina la dependencia de un servidor corporativo, manteniendo la funcionalidad prevista en el archivo \texttt{main.py}. Los objetivos espec\'ificos son documentar la arquitectura, las decisiones de dise\~no y las oportunidades de mejora en futuras iteraciones.

\section{Estructura del documento}
El documento se organiza en cinco cap\'itulos: el Cap.\ref{chap:introduccion} expone la motivaci\'on y alcance; el Cap.\ref{chap:estadoarte} sintetiza la literatura y normativa relevante; el Cap.\ref{chap:metodologia} detalla los materiales, m\'odulos y base de datos; el Cap.\ref{chap:resultados} muestra resultados preliminares y capturas funcionales; finalmente, el Cap.\ref{chap:conclusiones} presenta conclusiones y trabajo futuro.
