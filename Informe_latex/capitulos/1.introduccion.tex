\chapter{Introducci\'on}\label{chap:introduccion}

\section{Problema y prop\'osito}
En Lima Metropolitana y el Callao los reportes de adopci\'on, alertas de mascotas perdidas y denuncias de maltrato suelen fragmentarse en canales cerrados y publicaciones ef\'imeras. Esa dispersi\'on genera un ``laberinto'' de desinformaci\'on, repite esfuerzos y expone datos personales. La falta de un punto central para consolidar, filtrar y geolocalizar eventos incrementa el riesgo de estafas y retrasa la respuesta comunitaria.

``Patitas Seguras'' se concibi\'o como una aplicaci\'on de escritorio que reduce esa fricci\'on: concentra reportes, permite ubicarlos en mapa y mantiene un historial verificable. El proyecto busca transformar procesos que hoy demoran d\'ias en acciones de minutos mediante un flujo \emph{end-to-end} que une registro, consulta y localizaci\'on geogr\'afica.

\section{Objetivos generales y alcance}
El sistema se plantea como una herramienta integral de gesti\'on y cuidado animal, con m\'odulos especializados para adopci\'on, reportes de mascotas perdidas o encontradas, listas negras de maltrato y ranking comunitario de veterinarias. Opera sobre una pila tecnol\'ogica ligera (Python, Tkinter, Pillow, TkinterMapView y SQLite) para priorizar viabilidad en entornos sin conectividad permanente.

El alcance funcional se agrupa en tres l\'ineas: (a) gesti\'on de informaci\'on con operaciones CRUD, (b) geolocalizaci\'on y representaci\'on cartogr\'afica de eventos, y (c) mecanismos comunitarios de prevenci\'on mediante reputaci\'on y alertas. El objetivo operativo es ofrecer cat\'alogos filtrables, mapas interactivos y persistencia local con criterios claros de ordenamiento y trazabilidad.

\section{Estructura del documento}
El documento se organiza en cinco cap\'itulos: el Cap.\ref{chap:introduccion} expone el contexto y alcance; el Cap.\ref{chap:estadoarte} sintetiza la literatura y normativa relevante; el Cap.\ref{chap:metodologia} detalla arquitectura, modelo de datos, m\'odulos y criterios de calidad; el Cap.\ref{chap:resultados} resume implementaciones y hallazgos; finalmente, el Cap.\ref{chap:conclusiones} plantea conclusiones y una hoja de ruta futura.
