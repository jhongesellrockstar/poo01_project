\chapter{Estado del arte y contexto regional}\label{chap:estadoarte}

\section{Normativa y planes nacionales}
La Ley N\textdegree{} 30407 establece el marco peruano de protecci\'on y bienestar animal, obligando a denunciar el maltrato y promoviendo la tenencia responsable \citep{ley30407}. Complementariamente, el Servicio Nacional Forestal y de Fauna Silvestre (SERFOR) public\'o en 2023 un diagn\'ostico que subraya la falta de registros unificados de mascotas y la necesidad de articular a gobiernos locales \citep{serfor2023diagnostico}. Estas referencias justifican el componente de listas negras y la georreferenciaci\'on de reportes.

\section{Situaci\'on en Lima y Callao}
Estudios locales indican un aumento de denuncias por maltrato y sobrepoblaci\'on canina en distritos populosos de Lima y el Callao, lo que incrementa el riesgo de rabia y otras zoonosis \citep{ops2022rabia}. La Municipalidad Metropolitana de Lima emiti\'o ordenanzas que promueven la identificaci\'on de mascotas, pero la adopci\'on de herramientas digitales sigue siendo limitada \citep{ordenanzaLima}. En este escenario, una aplicaci\'on de escritorio con base SQLite facilita pilotos en laboratorios acad\'emicos y cabinas municipales sin conectividad permanente.

\section{Referentes tecnol\'ogicos}
Existen iniciativas de mapeo ciudadano basadas en \textit{open data} y sistemas ligeros (por ejemplo, dashboards de dengue y COVID-19) que inspiran el uso de \texttt{tkintermapview} y esquemas relacionales compactos \citep{whoZero2023}. La propuesta adopta patrones de arquitectura MVC y refuerza la trazabilidad mediante cat\'alogos de usuarios y auditor\'ia de votos.
