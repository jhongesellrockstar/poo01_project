\chapter{Materiales y m\'etodos}\label{chap:metodologia}

\section{Recursos de software}
El sistema se construy\'o con Python 3.11 y bibliotecas estables de escritorio: \texttt{tkinter} para la interfaz, \texttt{Pillow} para tratamiento de im\'agenes y \texttt{tkintermapview} para mapas interactivos. La persistencia recae en \texttt{sqlite3} con un archivo local \texttt{data.db} que elimina dependencias de servidor y permite replicar el prototipo en cabinas o laboratorios sin conectividad. Este enfoque prioriza viabilidad y portabilidad, manteniendo una experiencia fluida mediante estructuras en memoria.

\section{Arquitectura modular}
El dise\~no sigue un esquema monol\'itico modular con tres capas diferenciadas. La capa de presentaci\'on define la ventana principal, el dashboard y las pantallas de cada m\'odulo (adopci\'on, perdidos/encontrados, blacklist y ranking), reutilizando componentes como tarjetas, formularios y mapas. La l\'ogica de negocio concentra reglas de filtrado, validaci\'on b\'asica y c\'alculo de promedios de estrellas, mientras que la capa de datos inicializa tablas, carga cat\'alogos a memoria y gestiona inserciones o eliminaciones controladas. El flujo global arranca con la preparaci\'on del entorno y carga de datos, pasa por navegaci\'on de m\'odulos en el dashboard y concluye con commits y cierre seguro de la conexi\'on.

\section{Modelo de datos y persistencia}
SQLite se eligi\'o como motor embebido para lograr persistencia real sin despliegues adicionales. El esquema considera tablas para adopci\'on, veterinarias, listas negras (personas y establecimientos), mascotas perdidas y votos. Al iniciar la aplicaci\'on se ejecutan consultas \texttt{SELECT *} que poblan listas en memoria (por ejemplo, \texttt{db\_adopcion}, \texttt{db\_perdidos} o \texttt{db\_veterinarias}) para renderizar tarjetas con latencia m\'inima. Las operaciones de escritura validan campos obligatorios, usan sentencias parametrizadas y actualizan tanto la base como las estructuras en memoria antes de confirmar con \texttt{commit}. Se contempla eliminaci\'on controlada (p. ej., \texttt{DELETE FROM mascotas\_perdidas WHERE id = ?}) y se destaca la necesidad futura de llaves for\'aneas, \textit{checksums} de im\'agenes y auditor\'ia de acciones.

\section{Experiencia de usuario y navegaci\'on}
La UI se trata como componente de ingenier\'ia: la identidad visual y los componentes reutilizables reducen errores y aceleran la decisi\'on. La pantalla inicial presenta un dashboard con accesos directos a cada capacidad, y la navegaci\'on superior limpia y reemplaza paneles para mantener una sola ventana principal. Las tarjetas muestran fotograf\'ia y datos clave; los formularios incluyen controles estandarizados y los mapas permiten marcar coordenadas con un clic. La retroalimentaci\'on visual confirma env\'io de solicitudes o registros para evitar duplicados.

\section{M\'odulo de adopci\'on}
El cat\'alogo de adopci\'on convierte publicaciones dispersas en una vista filtrable por especie y etapa de vida. Cada tarjeta prioriza imagen, nombre y etiquetas de salud, mientras que la vista detalle habilita la acci\'on ``Quiero adoptar'' mediante un formulario b\'asico de contacto. El filtrado se realiza en memoria con complejidad lineal, lo que permite refrescos inmediatos y facilita la extensi\'on hacia nuevos atributos como distrito o tama\~no.

\section{M\'odulo de mascotas perdidas o encontradas}
Este m\'odulo evita la ``ceguera geogr\'afica'' al registrar puntos exactos en el mapa a trav\'es de \texttt{tkintermapview}. Cada reporte incluye estado (perdido/encontrado), descripci\'on, datos de contacto, foto y coordenadas, y se muestra como tarjeta con acceso a detalle y mapa. Se admite eliminaci\'on controlada por identificador durante la fase de prototipo, con la recomendaci\'on de evolucionar hacia estados (p. ej., resuelto) para preservar trazabilidad.

\section{M\'odulo de listas negras}
La blacklist funciona como mecanismo de prevenci\'on comunitaria. Para personas denunciadas se registran identificadores, descripci\'on, evidencia y ubicaci\'on; para veterinarias denunciadas se almacenan nombre, motivo y coordenadas. Ambos flujos generan tarjetas y mapas que permiten identificar puntos cr\'iticos. Se se\~nala el riesgo \'etico y legal de este tipo de registros y se proponen controles como estados de validaci\'on, moderaci\'on y pol\'iticas de uso claras.

\section{M\'odulo de ranking de veterinarias}
El ranking establece un sistema de reputaci\'on basado en estrellas y comentarios. Las tablas \texttt{veterinarias} y \texttt{votos} se vinculan para recalcular promedios cada vez que se inserta una rese\~na. La interfaz ordena las tarjetas por calificaci\'on promedio y muestra detalle con ubicaci\'on en mapa. Se reconocen riesgos de manipulaci\'on y se sugiere controlar duplicidad de votos, registrar identidad y moderar lenguaje en versiones futuras.

\section{Seguridad, integridad y calidad}
Los formularios validan campos obligatorios, rangos de coordenadas y formatos de tel\'efono, mientras que las operaciones en base usan sentencias parametrizadas y clausuras seguras para evitar corrupci\'on del archivo. Se recomienda incluir manejo expl\'icito de errores (im\'agenes, apertura de base, escritura) con mensajes claros y bit\'acoras. Para mitigar riesgos de difamaci\'on o exposici\'on de datos sensibles, el sistema debe mostrar advertencias visibles y adoptar moderaci\'on antes de publicar reportes.

\section{Plan de evoluci\'on y pruebas}
El roadmap contempla cuatro fases: (1) refactorizar en paquetes y reforzar validaciones y logging; (2) incorporar roles, estados de reporte y auditor\'ia para confiabilidad comunitaria; (3) habilitar sincronizaci\'on y API REST cuando sea necesario escalar; y (4) sumar anal\'iticas y visualizaciones avanzadas. Las pruebas sugeridas incluyen suites unitarias para la capa de datos y reglas de dominio, pruebas de integraci\'on sobre flujos de navegaci\'on y checklist visual por pantalla, adem\'as de empaquetado con PyInstaller para despliegue en escritorio.
